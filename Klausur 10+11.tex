\input{header.tex}
\usepackage{paralist}
\begin{document}

\maketitle

Dieser Text ist unter der
\href{http://creativecommons.org/licenses/by-nc/4.0/}{Creative Commons CC BY-NC 4.0}
Lizenz veröffentlicht.

\textcolor{red}{%
    Ich erhebe keinen Anspruch auf Vollständigkeit oder Richtigkeit. Falls ihr
    Fehler findet oder etwas fehlt, dann meldet euch bitte über den
    Emailkontakt.
}


\tableofcontents

\newpage


\section{Klausur 10}

\subsection{Aufgabe 1}

Bei dieser Aufgabe bin ich mir nicht sicher, ob die jeweiligen Lösungen korrekt sind. Wenn jemand begründet eine bessere hat, dann soll er sich melden.

\hfill\\

\begin{center}
\begin{tabular}{c|c|c|c|c|c|c|c|c|c}
a) & b) & c) & d) & e) & f) & g) & h) & i) & j) \\ 
\hline 
F & F & R & F & F & R & F & R & R & F \\ 
\end{tabular} 

\end{center}


\subsection{Aufgabe 2}

\subsubsection*{a)}

mechanisch, thermisch, optisch, akustisch

\subsubsection*{b)}

\begin{figure}[h]
\centering
\includegraphics[scale=0.6]{A2b.png}
\caption{$F_A$: Auftriebskraft, $F_W$: Widerstandskraft, $F_G$: Gewichtskraft}
\end{figure}

Der zu messende Luftstrom wird von unten in die Apparatur eingeleitet und drückt den Schwebekörper nach oben. Nachdem sich der Schwebekörper auf eine konstante Höhe eingependelt hat, kann man die Durchflussmenge anhand der Höhe bestimmen.

\subsubsection*{c)}

\begin{align*}
\dot{V} &= \frac{1}{\sqrt{c_w}} \cdot \left( A_R - A_S \right) \cdot \sqrt{\frac{2 \cdot g \cdot V_s \cdot \left( \rho_s - \rho_m \right)}{\rho_m \cdot A_s}}
\end{align*}


\subsubsection*{d)}


\begin{align*}
\text{Gase:} \qquad &\unit[1]{l/h} - \unit[2000]{m^3/h} \\
\text{Flüssigkeiten:} \qquad &\unit[100]{ml/h} - \unit[100]{m^3/h} \\
\end{align*}

\subsubsection*{e)}

\begin{itemize}
\item Ultraschall Laufzeitverfahren
\item Ultraschall Doppler Verfahren
\item ??
\end{itemize}


\subsubsection*{f)}

\begin{figure}[h]
\centering
\includegraphics[scale=0.5]{A2f.png}
\end{figure}


\subsection{Aufgabe 3}

\subsubsection*{a)}

\[ R(\nu) = R_0 \cdot \left( 1 + \alpha \nu \right) \]

$R_0$: Widerstand bei $\unit[0]{^\circ C}$ \\
$\alpha$: Temperaturkoeffizient


\subsubsection*{b)}

\begin{align*}
U(60) &= 100 \cdot \left( 1 + 3,85 \cdot 10^{-3} \cdot 60 \right) \cdot 0,001 = \unit[123,1]{mV} \\
U(100) &= 100 \cdot \left( 1 + 3,85 \cdot 10^{-3} \cdot 100 \right) \cdot 0,001 = \unit[138,5]{mV}
\end{align*}


\subsubsection*{c)}

DIN A, bei DIN B ist die Abweichung bereits zu groß.


\subsubsection*{d)}

Wir berechnen zunächst die Genauigkeit bei $\unit[60]{^\circ C}$:

\begin{align*}
\Delta \nu &= 0,15 + 0,002 \cdot 60 = \unit[0,27]{^\circ C}
\intertext{Die Gleichung für den Widerstand müssen wir nun differenzieren:}
\frac{\p R}{\p \nu} &= R_0 \cdot \alpha
\intertext{Zusammen mit der Temperaturabweichung können wir nun die Widerstandsabweichung bestimmen:}
\Delta R &= R_0 \cdot \alpha \cdot \nu = 100 \cdot 3,85 \cdot 10^{-3} \cdot 0,27 = \unit[0,104]{\Omega}
\end{align*}

Bei $\unit[60]{^\circ C}$ darf der Widerstand als in den Grenzen von $\unit[123 \pm 0,104]{\Omega}$ schwanken.


\subsubsection*{e)}

Da wir mit Vierleitermesstechnik messen, spielen die Widerstände der Leitungen keine Rolle! Man misst also die berechneten $\unit[123,1]{mV}$.


\subsubsection*{f)}

\begin{align*}
P &= R \cdot I^2 = 123,1 \cdot 0,001^2 = \unit[1,23 \cdot 10^{-4}]{W} \\
\Rightarrow \Delta \nu &= 0,16 \cdot 0,123 = \unit[0,0196]{^\circ C} 
\end{align*}




\end{document}
