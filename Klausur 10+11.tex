\input{header.tex}
\usepackage{paralist}
\begin{document}

\maketitle

Dieser Text ist unter der
\href{http://creativecommons.org/licenses/by-nc/4.0/}{Creative Commons CC BY-NC 4.0}
Lizenz veröffentlicht.

\textcolor{red}{%
    Ich erhebe keinen Anspruch auf Vollständigkeit oder Richtigkeit. Falls ihr
    Fehler findet oder etwas fehlt, dann meldet euch bitte über den
    Emailkontakt.
}


\tableofcontents

\newpage


\section{Klausur 10}

\subsection{Aufgabe 1}

Bei dieser Aufgabe bin ich mir nicht sicher, ob die jeweiligen Lösungen korrekt sind. Wenn jemand begründet eine bessere hat, dann soll er sich melden.

\hfill\\

\begin{center}
\begin{tabular}{c|c|c|c|c|c|c|c|c|c}
a) & b) & c) & d) & e) & f) & g) & h) & i) & j) \\ 
\hline 
F & F & R & F & F & R & F & R & R & F \\ 
\end{tabular} 

\end{center}


\subsection{Aufgabe 2}

\subsubsection*{a)}

mechanisch, thermisch, optisch, akustisch

\subsubsection*{b)}

\begin{figure}[h]
\centering
\includegraphics[scale=0.6]{A2b.png}
\caption{$F_A$: Auftriebskraft, $F_W$: Widerstandskraft, $F_G$: Gewichtskraft}
\end{figure}

Der zu messende Luftstrom wird von unten in die Apparatur eingeleitet und drückt den Schwebekörper nach oben. Nachdem sich der Schwebekörper auf eine konstante Höhe eingependelt hat, kann man die Durchflussmenge anhand der Höhe bestimmen.


\end{document}
